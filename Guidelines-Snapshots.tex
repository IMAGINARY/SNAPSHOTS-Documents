\documentclass[a4paper,12pt]{article}
\usepackage[foot=60pt,head=26.6pt]{geometry}
\usepackage{graphicx}
\usepackage{pstricks}
\usepackage{afterpage}
\usepackage{fancyhdr}
\pagestyle{fancy}
\renewcommand{\headrulewidth}{0pt}
\lhead{}
\chead{}
\rhead{}
\lfoot{}
\cfoot{\includegraphics[width=\textwidth]{logos-snapshots}}
\rfoot{}

\fancypagestyle{plain}{
\fancyhf{} % clear all header and footer fields
\fancyfoot[C]{\bfseries \thepage} % except the center
\renewcommand{\headrulewidth}{0pt}
\renewcommand{\footrulewidth}{0pt}}

\usepackage{hyperref}

\title{Snapshots from Oberwolfach}
\date{}
\author{Guidelines}

\begin{document}
%TODO: Logos der Tschira-Stiftung, der Oberwolfach-Stiftung, des MFO sch\"on einbinden
\maketitle
\thispagestyle{fancy}
\afterpage{\chead{\thepage}}

Thank you for considering to write a snapshot of modern mathematics. The snapshot project is designed to promote the understanding and appreciation of modern mathematics and mathematical research in the general public world-wide. Please find some useful information below. Do not hesitate to contact us in case you have any questions or comments.

\subsubsection*{\it What is a snapshot?}
A {\it snapshot of modern mathematics from Oberwolfach} is a short text (approx.\ 5-8 A5 sized pages) that explains a mathematical problem or idea that is related to a scientific program at the MFO. It is targeted at a general audience consisting of advanced high school and undergraduate students, mathematics teachers, science journalists, and other individuals interested in modern mathematics. They can be written in English or German. 

\subsubsection*{\it Who can write a snapshot?}
Any participant or group of participants in the scientific programs of the MFO is invited to write a snapshot of modern mathematics in consent with the organizers of the respective program. Previous experience in mathematics communication with the general public (e.g.~with high school students, science journalists, teachers, etc.) is preferable. If you are interested in writing a snapshot, please contact the organizers of your program.

\subsubsection*{\it How will the snapshots be edited?}
The organizers of the scientific programs of the MFO will nominate up to three participants or groups of participants as authors of snapshots. The nominated authors will receive an easy to handle \LaTeX~template for the snapshots from the MFO. Each snapshot will be communicated to the snapshot editors by the corresponding organizers. It will then be edited, formatted, and possibly illustrated in consent with the author(s) by our junior editors. Upon approval by the director of the MFO, Gerhard Huisken, they will be uploaded to the mathematics communication platform \url{www.imaginary.org}.

\subsubsection*{\it Who will read your snapshot?}
The open source and participative platform \url{www.imaginary.org} allows anyone to upload and download material (picture galleries, software packages, explanatory texts, videos etc.) related to mathematics. It is currently available in English, German, and Spanish and will soon be available in other languages, too. The platform \url{www.imaginary.org} is used widely by teachers who want to enhance their mathematics courses, by museums, mathematics departments and research institutes who plan exhibitions on mathematics, by individuals who want to learn more about mathematics, etc. It was launched in February 2013 by the MFO and is funded by the Klaus Tschira Foundation.

\subsubsection*{\it What will happen to your snapshot in the future?}
The snapshots will remain accessible on the platform \url{www.imaginary.org} as PDFs. The IMAGINARY community is very active in translating material to other languages and uploading their translations back to the platform for anyone's use. Also, the media as well as museums and exhibitors are keen to use material from the platform to present modern mathematical research to the general public.

We are currently developing license options that will allow you to choose if you permit the IMAGINARY community to translate your snapshot to other languages with or without contacting you. Also, the license options will allow you to specify which further usage of your snapshot you agree to.

\subsubsection*{\it What are the goals of the snapshot project?}
The snapshot project is part of the project {\it Oberwolfach meets IMAGINARY} for which the MFO has recently been awarded a major grant by the Klaus Tschira Foundation and the Oberwolfach Foundation. In this project, we plan to create and collect high quality mathematics communication content for the platform \url{www.imaginary.org} at the MFO. Our goal is to promote the understanding and appreciation of modern mathematics and mathematical research in the general public world-wide and to provide interested individuals as well as exhibitors, teachers, and science journalists with accessible insights into modern mathematics.
\newpage
\subsubsection*{\it Hints and suggestions for writing a superb snapshot:}
When planning and writing your snapshot of modern mathematics, it is very important to keep your future audience in mind: the youngest readers will be advanced high school students who do not yet have experience with abstract mathematics. Also, your readers might not be native speakers of English. You can therefore not assume familiarity with many mathematical concepts; as a rule of thumb, it might help to recall what you knew when you were in the earliest stages of your mathematical interest.

While we explicitly ask you to write about modern mathematics related to the research focus of your scientific program at the MFO, we understand that it will hardly be possible to fully explain the ideas being discussed in your field at the level of the readers. However, we hope that you will be able to identify one or a few interesting aspects or general themes that can be explained. Correctness and completeness of your exhibition will be less central for your readers than understandability and accessibility. Maybe you can find a good metaphor for an idea you want to explain? Or can you think of an everyday analog of some concept or phenomenon from your field? Again, you might find it useful to picture someone specific such as your children, your favorite high school teacher, or your first year students and try to get across to them what your research field is all about. What would fascinate or surprise them?

Feel free to add to your text any illustrations, photographs, web links or references that could enhance your explanations. If you use illustrations or photographs produced by others -- in particular if you download them from the world wide web, please make sure that you have the relevant copyrights and give all necessary credits.\\[-1ex]

\noindent {\bf Contact:}\\[1ex]
\begin{tabular}{ll}
Senior Editor:\phantom{s} Dr.~Carla Cederbaum &\href{mailto:cederbaum@mfo.de}{\nolinkurl{cederbaum@mfo.de} }\\
Junior Editors: Sophia Jahns \&  &\href{mailto:junior-editors@mfo.de}{\nolinkurl{junior-editors@mfo.de} }\\
\phantom{Junior Editors: }Lea Renner
\end{tabular}
\vspace{2ex}

\noindent  {\bf Oberwolfach meets IMAGINARY}\\[1ex]
\begin{tabular}{ll}
Prof.~Dr.~Gerhard Huisken & (Director)\\
Prof.~Dr.~Gert-Martin Greuel\quad\quad\quad\quad & (Scientific Advisor)\\
Dr.~Andreas Daniel Matt & (Scientific Coordinator)\\
Dr.~Carla Cederbaum & (Scientific Coordinator)\\
\quad&\\[-1ex]
\end{tabular}

\noindent {\bf Mathematisches Forschungsinstitut Oberwolfach gGmbH}\\[1ex]
\begin{tabular}{ll}
Schwarzwaldstr. 9-11&\,\,\,{\it phone:} +49-(0)7834/979-0\\
77709 Oberwolfach-Walke\quad\quad\quad\quad&\,\,\,{\it fax:} +49-(0)7834/979-38\\
Germany&\,\,\,\url{http://www.mfo.de}
\end{tabular}
\end{document}